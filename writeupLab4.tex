\documentclass{article}\usepackage[]{graphicx}\usepackage[]{color}
%% maxwidth is the original width if it is less than linewidth
%% otherwise use linewidth (to make sure the graphics do not exceed the margin)

\makeatletter
\def\maxwidth{ %
  \ifdim\Gin@nat@width>\linewidth
    \linewidth
  \else
    \Gin@nat@width
  \fi
}
\makeatother

\usepackage{framed}
\makeatletter

\definecolor{fgcolor}{rgb}{0.345, 0.345, 0.345}
\newcommand{\hlnum}[1]{\textcolor[rgb]{0.686,0.059,0.569}{#1}}%
\newcommand{\hlstr}[1]{\textcolor[rgb]{0.192,0.494,0.8}{#1}}%
\newcommand{\hlcom}[1]{\textcolor[rgb]{0.678,0.584,0.686}{\textit{#1}}}%
\newcommand{\hlopt}[1]{\textcolor[rgb]{0,0,0}{#1}}%
\newcommand{\hlstd}[1]{\textcolor[rgb]{0.345,0.345,0.345}{#1}}%
\newcommand{\hlkwa}[1]{\textcolor[rgb]{0.161,0.373,0.58}{\textbf{#1}}}%
\newcommand{\hlkwb}[1]{\textcolor[rgb]{0.69,0.353,0.396}{#1}}%
\newcommand{\hlkwc}[1]{\textcolor[rgb]{0.333,0.667,0.333}{#1}}%
\newcommand{\hlkwd}[1]{\textcolor[rgb]{0.737,0.353,0.396}{\textbf{#1}}}%



\usepackage{framed}
\makeatletter
\newenvironment{kframe}{%
 \def\at@end@of@kframe{}%
 \ifinner\ifhmode%
  \def\at@end@of@kframe{\end{minipage}}%
  \begin{minipage}{\columnwidth}%
 \fi\fi%
 \def\FrameCommand##1{\hskip\@totalleftmargin \hskip-\fboxsep
 \colorbox{shadecolor}{##1}\hskip-\fboxsep
     % There is no \\@totalrightmargin, so:
     \hskip-\linewidth \hskip-\@totalleftmargin \hskip\columnwidth}%
 \MakeFramed {\advance\hsize-\width
   \@totalleftmargin\z@ \linewidth\hsize
   \@setminipage}}%
 {\par\unskip\endMakeFramed%
 \at@end@of@kframe}
\makeatother

\definecolor{shadecolor}{rgb}{.97, .97, .97}
\definecolor{messagecolor}{rgb}{0, 0, 0}
\definecolor{warningcolor}{rgb}{1, 0, 1}
\definecolor{errorcolor}{rgb}{1, 0, 0}
\newenvironment{knitrout}{}{} % an empty environment to be redefined in TeX

\usepackage{alltt}
\usepackage{fancyhdr}
\pagestyle{fancy}
\setlength{\parskip}{\smallskipamount}
\setlength{\parindent}{0pt}
\usepackage{amsthm}
\usepackage{amsmath}
\usepackage{wrapfig}
\usepackage{graphicx}
\usepackage{float}
\usepackage[margin=1in]{geometry}
\IfFileExists{upquote.sty}{\usepackage{upquote}}{}

\newtheorem{corollary}{Corollary}[section]
\newtheorem{theorem}{Theorem}
\newtheorem{example}{Example}
\newtheorem{question}{Question}
\newtheorem{remark}{Remark}[section]
\newtheorem{conjecture}{Conjecture}
\newtheorem{proposition}{Proposition}[section]
\newtheorem{definition}{Definition}[section]


\begin{document}


\title{Lab 4-Binary Classifier\\
Stat 215A, Fall 2014}

\author{PLEASE WRITE YXiang (Lisha) Li}

\maketitle


\section{Introduction}
Blah blah...

\section{EDA}
\subsection{Densities of the three most important features as selected by paper}



\begin{figure}[h]
\minipage{0.32\textwidth}
  \includegraphics[width=\linewidth]{NDAI1.png}
  \caption{NDAI density Image 1}\label{}
\endminipage\hfill
\minipage{0.32\textwidth}
  \includegraphics[width=\linewidth]{NDAI2.png}
  \caption{NDAI density Image 2}\label{}
\endminipage\hfill
\minipage{0.32\textwidth}%
  \includegraphics[width=\linewidth]{NDAI3.png}
  \caption{NDAI density Image 3}\label{}
\endminipage
\end{figure}

\begin{figure}[h]
\minipage{0.32\textwidth}
  \includegraphics[width=\linewidth]{SD1.png}
  \caption{SD density Image 1}\label{}
\endminipage\hfill
\minipage{0.32\textwidth}
  \includegraphics[width=\linewidth]{SD2.png}
  \caption{SD density Image 2}\label{}
\endminipage\hfill
\minipage{0.32\textwidth}%
  \includegraphics[width=\linewidth]{SD3.png}
  \caption{SD density Image 3}\label{}
\endminipage
\end{figure}


\begin{figure}[h]
\minipage{0.32\textwidth}
  \includegraphics[width=\linewidth]{CORRvsNDAI.png}
  \caption{CORR vs. NDAI Plot of Image 1}\label{}
\endminipage\hfill
\minipage{0.32\textwidth}
  \includegraphics[width=\linewidth]{CORRvsSD.png}
  \caption{CORR vs. SD Plot of Image 1}\label{}
\endminipage\hfill
\minipage{0.32\textwidth}%
  \includegraphics[width=\linewidth]{NDAIvsSD.png}
  \caption{NDAI vs. SD Plot of Image1}\label{}
\endminipage
\end{figure}

\begin{figure}[h]
\minipage{0.32\textwidth}
  \includegraphics[width=\linewidth]{CORR1.png}
  \caption{CORR density Image 1}\label{}
\endminipage\hfill
\minipage{0.32\textwidth}
  \includegraphics[width=\linewidth]{CORR2.png}
  \caption{CORR density Image 2}\label{}
\endminipage\hfill
\minipage{0.32\textwidth}%
  \includegraphics[width=\linewidth]{CORR3.png}
  \caption{CORR density Image 3}\label{}
\endminipage
\end{figure}


\section{Modeling}

\subsection{LDA}

\subsection{QDA}

\begin{figure}[h]
\minipage{0.49\textwidth}
  \includegraphics[width=\linewidth]{QDArocCV.png}
  \caption{Response Operator Curve for 12-fold Cross-valudation of QDA}\label{}
\endminipage\hfill
\minipage{0.49\textwidth}
  \includegraphics[width=\linewidth]{AUCcvQDA.png}
  \caption{AUC for QDA Classifiers}\label{rocauc}
\endminipage\hfill
\end{figure}

Cross-validation for QDA revealed that while the method works extremely well in some cases, producing AUC scores of almost 1, it sometimes fails to perform better than even the theoretical random classifier.  In particular, the classifier does not seem to be good at discerning snow from clouds in regions where there are many dark pixels.  For example, during cross-validation, the watery bottom-left quadrant of image 2 and ridge-ridden left edge of image 3 poses significant problems to our classifiers.

\subsection{Logit/Probit}

\subsection{Random Forest}

For random forest, as in all the other classifiers, we divided the three images into equal sized quadrants (2X2) rectangles in order to do 12 fold validation on the dataset.  That is, for each iteration of the validation, we dropped one of the quadrants as a test set, and trained on the remaining 11 quadrants.  Keeping the images segments disjoint and continuous ensured that our models were picking up on `higher' level structure of the dataset, and not the continuous variation of neighbouring pixels.  We also trained on each image and tested on the remaining two.  To test convergence, one of the things we did was increase the training set from including 1 quadrant, to including 2 quadrants, up to including 11 quadrants (using the complement as the test set).  For each of the 3 aforementioned classes of training, we trained on a range of forest sizes, from 2 trees to 50.     Here are the results: 

Finally, this entire set of training models was done with all the features, and then restricted to only SD, CORR and NDAI.  These three were particularly chosen because their GiniImportance was consistently ranked at least 2fold above the next highest in all the cross validations. As we also saw with the random forest model that just used SD, CORR, NDAI, it did not fare poorly compared to training the forests on all 9 predictors.  

\begin{figure}[h]
\minipage{0.32\textwidth}
  \includegraphics[width=\linewidth]{AUCconverge.png}
  \caption{Smoothed convergence of AUC for growing training set 50 trees and 3 features}\label{}
\endminipage\hfill
\minipage{0.32\textwidth}
  \includegraphics[width=\linewidth]{CVAUC.png}
  \caption{AUC of test set in each fold with 50 trees and 3 features}\label{}
\endminipage\hfill
\minipage{0.32\textwidth}%
  \includegraphics[width=\linewidth]{imageAUC.png}
  \caption{AUC of the three images with 50 trees and 3 features}\label{}
\endminipage
\end{figure}


{\bf Cross Validation ROC curves}

\begin{figure}[h]
\minipage{0.32\textwidth}
  \includegraphics[width=\linewidth]{ROC_block1.pdf}
  \caption{ROC fold 1}\label{}
\endminipage\hfill
\minipage{0.32\textwidth}
  \includegraphics[width=\linewidth]{ROC_block2.pdf}
  \caption{ROC fold 2}\label{}
\endminipage\hfill
\minipage{0.32\textwidth}%
  \includegraphics[width=\linewidth]{ROC_block3.pdf}
  \caption{ROC fold 3}\label{}
\endminipage
\end{figure}

\begin{figure}[h]
\minipage{0.32\textwidth}
  \includegraphics[width=\linewidth]{ROC_block4.pdf}
  \caption{ROC fold 4}\label{}
\endminipage\hfill
\minipage{0.32\textwidth}
  \includegraphics[width=\linewidth]{ROC_block5.pdf}
  \caption{ROC fold 5}\label{}
\endminipage\hfill
\minipage{0.32\textwidth}%
  \includegraphics[width=\linewidth]{ROC_block6.pdf}
  \caption{ROC fold 6}\label{}
\endminipage
\end{figure}

\begin{figure}[h]
\minipage{0.32\textwidth}
  \includegraphics[width=\linewidth]{ROC_block8.pdf}
  \caption{ROC fold 8}\label{}
\endminipage\hfill
\minipage{0.32\textwidth}
  \includegraphics[width=\linewidth]{ROC_block9.pdf}
  \caption{ROC fold 9}\label{}
\endminipage\hfill
\minipage{0.32\textwidth}%
  \includegraphics[width=\linewidth]{ROC_block10.pdf}
  \caption{ROC fold 10}\label{}
\endminipage
\end{figure}

\begin{figure}[h]
\minipage{0.32\textwidth}
  \includegraphics[width=\linewidth]{ROC_block11.pdf}
  \caption{ROC fold 11}\label{}
\endminipage\hfill
\minipage{0.32\textwidth}%
  \includegraphics[width=\linewidth]{ROC_block12.pdf}
  \caption{ROC fold 12}\label{}
\endminipage
\end{figure}

{\bf ROC curves for cross validation between images} 

\begin{figure}[h]
\minipage{0.32\textwidth}
  \includegraphics[width=\linewidth]{ROC_image1.pdf}
  \caption{Trained on image1}\label{}
\endminipage\hfill
\minipage{0.32\textwidth}
  \includegraphics[width=\linewidth]{ROC_image4.pdf}
  \caption{Trained on image2}\label{}
\endminipage\hfill
\minipage{0.32\textwidth}%
  \includegraphics[width=\linewidth]{ROC_image8.pdf}
  \caption{Trained on image3}\label{}
\endminipage
\end{figure}


\section{Reproducibility}

{\bf How we organized out code and github repo}


 \begin{thebibliography}{1}

\bibitem{notes} Ben-Hur, A., Elisseeff, A., Guyon, I.: A stability based method for discovering structure in clustered data. In: Pacific Symposium on Biocomputing, pp. 6�17.   
\end{thebibliography}


\end{document}